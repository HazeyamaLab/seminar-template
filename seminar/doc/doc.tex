\documentclass[dvipdfmx]{doc}
\usepackage[dvipdfmx]{graphicx} % 図のパッケージ
\usepackage{multirow} % 表で複数行使用するため
\usepackage{url} % 参考文献にurlを入れるため

\title{研究タイトル} % 論文タイトル
\etitle{title \LaTeX} % 英語タイトル
\affiliation{櫨山研究室}
\studentid{X00-0000}
\author{ 学芸 太郎} % 学籍番号&名前
\date{20xx/01/01} %ゼミの日付
% \date{} %日付を載せない資料の時は中身を消す.

\begin{document}

\maketitle

\section{はじめに}

研究の背景や目的について記述しましょう.

\section{関連研究}

どの観点から関連研究を持ってきたかを記述しましょう.

\subsection{関連研究1}

引っ張ってきた関連研究について記述しましょう.
対象としている問題点・アプローチ・結果・考察を意識しましょう.

\subsection{本研究の位置付け}

関連研究と比較した上で,自分の研究の位置付けを記述しましょう.

\section{本研究のアプローチ}

具体的な研究のアプローチについて記述しましょう.
アプローチによってどんな結果が得られるかを考え,評価手法も考慮しながら述べると良いです.

\section{おわりに}

ここまでで述べたことをまとめながら,研究で達成したことについて記述しましょう.

\begin{thebibliography}{10}
\end{thebibliography}

\subsection*{前回の議論}
\begin{itemize}
    \item 箇条書きで前回の議論の内容を記述 \\
    応答する形で自分の考え・意見・進捗を記述する.
\end{itemize}

\subsection*{進捗}
\begin{itemize}
    \item 箇条書きで1ヶ月の進捗を記述する.
\end{itemize}

\subsection*{今後の予定}
\begin{itemize}
    \item 箇条書きで今後の予定について記述する.
\end{itemize}

\end{document}
